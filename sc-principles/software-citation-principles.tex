\documentclass[11pt, oneside]{amsart}
\pdfoutput=1

\usepackage{amsmath}
\usepackage{amssymb}

\usepackage{color}
\usepackage{dcolumn}
\usepackage{float}
\usepackage{graphicx}
\usepackage[utf8]{inputenc}
\usepackage[T1]{fontenc}
\usepackage{lmodern}
\usepackage{multirow}
\usepackage{rotating}
\usepackage{subfigure}
\usepackage{psfrag}
\usepackage{tabularx}
\usepackage[hyphens]{url}
\usepackage{wrapfig}
\usepackage{longtable}
\usepackage{verbatim}
\usepackage{booktabs,multicol}

% The following three lines are used for displaying footnote in tables.
\usepackage{footnote}
\makesavenoteenv{tabular}
\makesavenoteenv{table}

\usepackage{enumitem}
\setlist{leftmargin=7mm}

\usepackage[bookmarks, bookmarksopen, bookmarksnumbered]{hyperref}
\usepackage[all]{hypcap}
\urlstyle{rm}


\newcommand{\katznote}[1]{ {\textcolor{blue} { ***DSK: #1 }}}
\newcommand{\niemeyernote}[1]{ {\textcolor{orange} { ***KEN: #1 }}}



% 15 characters / 2.5 cm => 100 characters / line
% Using 11 pt => 94 characters / line
\setlength{\paperwidth}{216 mm}
% 6 lines / 2.5 cm => 55 lines / page
% Using 11pt => 48 lines / pages
\setlength{\paperheight}{279 mm}
\usepackage[top=2.5cm, bottom=2.5cm, left=2.5cm, right=2.5cm]{geometry}
% You can use a baselinestretch of down to 0.9
\renewcommand{\baselinestretch}{0.96}



\title{Software Citation Principles}

\author{}

\date{}

\begin{document}

\begin{abstract}
\end{abstract}

\maketitle


%%%%%%%%%%%%%%%%%%%%%%%%%%%%%%%%%%%%%%%%%%%%%%%%%%%%%%%%%%%%
\section{Introduction \& motivation}
\label{sec:intro}
%%%%%%%%%%%%%%%%%%%%%%%%%%%%%%%%%%%%%%%%%%%%%%%%%%%%%%%%%%%%

As the process of science has become increasingly digital, scientific outputs and products have
grown beyond simple papers and books to include software, data, and other electronic
components.  Scientific knowledge is embedded in these components.  And papers and books
themselves are also becoming increasingly digital, allowing them to executable
and reproducible.  As we move towards this future where science is performed in and recorded
as a variety of linked digital products, the characteristics and properties that developed for
books and papers need to be applied to all digital products.  Here, we are concerned specifically
with the citation of software products.

In the next section (\S\ref{sec:use_cases}), we provide some detailed context in which
software citation is important, by means of use cases.  In \S\ref{sec:related_work}, we
summarize and analyze a large amount of previous work and thinking in this area.  In
\S\ref{sec:principles}, we provide set of guiding principles for citation of software within
scholarly literature, other software dataset, or any other research object.  Finally,
in \S\ref{sec:examples} we give examples of how these software citation principles
could be applied, related back to the use cases in \S\ref{sec:use_cases}.

%%%%%%%%%%%%%%%%%%%%%%%%%%%%%%%%%%%%%%%%%%%%%%%%%%%%%%%%%%%%
\section{Use cases}
\label{sec:use_cases}
%%%%%%%%%%%%%%%%%%%%%%%%%%%%%%%%%%%%%%%%%%%%%%%%%%%%%%%%%%%%

\begin{table}[htbp]
\caption{Use cases for software citation, adapted from \cite{SC-Use-Cases}}
\centering
\scriptsize\setlength{\tabcolsep}{2.5pt}
\begin{tabular}{@{}l l c c c c c c c c@{}}
\toprule
 & & \multicolumn{6}{c}{Requirements} \\
 \cmidrule{3-10}
\multirow{2}{*}{Stakeholder} &	\multirow{2}{*}{Use\slash wants to} 	 &  Software  & \multirow{2}{*}{Author(s)} & \multirow{2}{*}{Version \#} & Release & Location\slash  & \multirow{2}{*}{DOI} & Indexed & \multirow{2}{*}{Role} \\
& & name &  &  &  date & repository &  & citations & \\
\midrule
Researcher & someone else's software for a paper	& \textbullet	& \textbullet & \textbullet & \textbullet & \textbullet & \textbullet & & \\
Researcher & someone else's software for new software& \textbullet	& \textbullet & \textbullet & \textbullet & \textbullet & \textbullet & & \\
Researcher & contribute to software 				& \textbullet	& \textbullet & \textbullet & \textbullet & \textbullet & \textbullet & & \textbullet \\
Researcher & find citations of software 			& \textbullet &   &   &   &   & \textbullet & \textbullet & \\
Researcher & credit for software development		& \textbullet & \textbullet &   & \textbullet & \textbullet & \textbullet & & \textbullet \\
Researcher & ``reproduce'' analysis 				& \textbullet &   & \textbullet & \textbullet & \textbullet & \textbullet & & \\
Researcher & benchmark software 					& \textbullet &   & \textbullet & \textbullet & \textbullet & \textbullet & & \\
Researcher & find software to implement task 		& \textbullet & \textbullet &   &   & \textbullet & \textbullet & \textbullet & \\
Publisher & publish software paper					& \textbullet & \textbullet & \textbullet & \textbullet & \textbullet & \textbullet & & \\
Publisher & publish papers that cite software 		& \textbullet & \textbullet & \textbullet & \textbullet & \textbullet & \textbullet & \textbullet & \\
Indexer & build catalog of software 				& \textbullet & \textbullet & \textbullet & \textbullet & \textbullet & \textbullet & \textbullet & \\
Domain group & build catalog of software 			& \textbullet & \textbullet & \textbullet & \textbullet & \textbullet & \textbullet &  & \\
Library\slash archive & build catalog of software 	& \textbullet & \textbullet & \textbullet & \textbullet & \textbullet & \textbullet &  & \\
Repository & show scientific impact of holdings 	& \textbullet &   &   &   &   & \textbullet & \textbullet & \\
Funder & show how funded software has been used		& \textbullet &   &   &   &   & \textbullet & \textbullet & \\
\bottomrule
\end{tabular}
\label{tab:use_cases}
\end{table}%

``Researcher'' includes academic researchers (e.g., postdoc, tenure-track faculty member) and research software engineers.
``Reproduce'' can mean reproduction, replication, verification, validation, repeatability, and\slash or utility.
Examples of indexers include Scopus, Web of Science, Google Scholar, and Microsoft Academic Search.
Domain groups include ASCL, bioCADDIE, CIG, etc.

Alternate metadata
\begin{itemize}
\item version number and release date: download date
\item location\slash respository: contact name\slash email if not publicly available
\end{itemize}




%%%%%%%%%%%%%%%%%%%%%%%%%%%%%%%%%%%%%%%%%%%%%%%%%%%%%%%%%%%%
\section{Related work}
\label{sec:related_work}
%%%%%%%%%%%%%%%%%%%%%%%%%%%%%%%%%%%%%%%%%%%%%%%%%%%%%%%%%%%%



%%%%%%%%%%%%%%%%%%%%%%%%%%%%%%%%%%%%%%%%%%%%%%%%%%%%%%%%%%%%
\section{Software citation principles}
\label{sec:principles}
%%%%%%%%%%%%%%%%%%%%%%%%%%%%%%%%%%%%%%%%%%%%%%%%%%%%%%%%%%%%

\katznote{The initial context of this section is liberally borrowed and loosely adapted from \cite{data-citation-principles}}

\begin{enumerate}
\item \textbf{Importance}:
Software should be considered a legitimate, citable product of research. Software citations should be accorded the same importance in the scholarly record as citations of other research objects, such as publications and data.
\item \textbf{Credit and Attribution}
\begin{enumerate}
\item Software citations should facilitate giving scholarly credit and normative and legal attribution to all contributors to the software, recognizing that a single style or mechanism of attribution may not be applicable to all software.
\item In scholarly literature and other products, whenever and wherever a claim relies upon software, the corresponding software should be cited.
\end{enumerate}
\item \textbf{Unique Identification}
A software citation should include a persistent method for identification that is machine actionable, globally unique, and widely used by a community.
\item \textbf{Access}
Software citations should facilitate access to the software themselves and to such associated metadata, documentation, data, and other materials, as are necessary for both humans and machines to make informed use of the referenced software.
\item \textbf{Persistence}
Unique identifiers, and metadata describing the software, and its disposition, should persist -- even beyond the lifespan of the software they describe.
\item \textbf{Specificity and Verifiability}
Software citations should facilitate identification of, access to, and verification of the specific software that support a claim. Citations or citation metadata should include information about provenance and fixity sufficient to facilitate verifying that the specific version and/or portion of software retrieved subsequently is the same as was originally cited.
\item \textbf{Interoperability and Flexibility}
Software citation methods should be sufficiently flexible to accommodate the variant practices among communities, but should not differ so much that they compromise interoperability of software citation practices across communities.
\end{enumerate}


%%%%%%%%%%%%%%%%%%%%%%%%%%%%%%%%%%%%%%%%%%%%%%%%%%%%%%%%%%%%
\section{Implementation examples}
\label{sec:examples}
%%%%%%%%%%%%%%%%%%%%%%%%%%%%%%%%%%%%%%%%%%%%%%%%%%%%%%%%%%%%

\bibliographystyle{abbrv}
\bibliography{software-citation-principles}


\end{document}